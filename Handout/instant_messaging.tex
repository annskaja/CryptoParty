\documentclass{tufte-handout}
%\geometry{showframe}% for debugging purposes -- displays the margins

\usepackage{amsmath}

% Set up the images/graphics package
\usepackage{graphicx}
\setkeys{Gin}{width=\linewidth,totalheight=\textheight,keepaspectratio}
\graphicspath{{graphics/}}

\title{Verschl�sseltes Instant Messaging}
\author[CryptoParty]{CryptoParty Dresden}
\date{6. Dezember 2012}  % if the \date{} command is left out, the current date will be used

% The following package makes prettier tables.  We're all about the bling!
\usepackage{booktabs}

% The units package provides nice, non-stacked fractions and better spacing
% for units.
\usepackage{units}
\usepackage[ngerman]{babel}
\usepackage[latin1]{inputenc}

% The fancyvrb package lets us customize the formatting of verbatim
% environments.  We use a slightly smaller font.
\usepackage{fancyvrb}
\fvset{fontsize=\normalsize}

% Small sections of multiple columns
\usepackage{multicol}

% Provides paragraphs of dummy text
\usepackage{lipsum}

% These commands are used to pretty-print LaTeX commands
\newcommand{\doccmd}[1]{\texttt{\textbackslash#1}}% command name -- adds backslash automatically
\newcommand{\docopt}[1]{\ensuremath{\langle}\textrm{\textit{#1}}\ensuremath{\rangle}}% optional command argument
\newcommand{\docarg}[1]{\textrm{\textit{#1}}}% (required) command argument
\newenvironment{docspec}{\begin{quote}\noindent}{\end{quote}}% command specification environment
\newcommand{\docenv}[1]{\textsf{#1}}% environment name
\newcommand{\docpkg}[1]{\texttt{#1}}% package name
\newcommand{\doccls}[1]{\texttt{#1}}% document class name
\newcommand{\docclsopt}[1]{\texttt{#1}}% document class option name

\begin{document}

\maketitle% this prints the handout title, author, and date

\begin{abstract}
\noindent Instant Messaging, beispielsweise �ber ICQ, ist standardm��ig unverschl�sselt. Das hei�t, jeder Lauscher auf der Leitung kann im Klartext lesen, welche Nachrichten Du schreibst und bekommst. Mit ein paar kleinen Vorkehrungen l�sst sich das aber leicht verhindern
\end{abstract}

\section{Off-the-Record Messaging (OTR)}\label{sec:page-layout}
OTR ist ein Protokoll, um Instant Messaging zu verschl�sseln. Doch dar�berhinaus bietet eine mit OTR verschl�sselte Nachricht auch Abstreitbarkeit, das hei�t, dass Nachrichten nur zum Zeitpunkt der Unterhaltung zwischen den Teilnehmern nachweisbar dem Sender zugeordnet werden k�nnen. Die Signierung ist nur gegen�ber dem Empf�nger der Nachricht glaubw�rdig, der Empf�nger selbst k�nnte n�mlich nach Erhalt der Nachricht die Signatur einfach nachmachen, kann so also gegen�ber Dritten nicht glaubhaft machen, dass Du ihm diese Nachricht geschrieben hast und nicht er selbst. Au�erdem ist die Kommunikation via OTR auch bei Verlust eines privaten Schl�ssels folgenlos, da dieser nur zur Erzeugung der vor�bergehenden privaten Schl�ssel dient, mit dem die einzelnen Nachrichten dann signiert werden.

\subsection{Benutzung von OTR}
Eine der einfachsten Methoden, OTR zu benutzen, ist besipielsweise mit Pidgin\sidenote{https://pidgin.im/}. Pidgin ist ein freier Instant Messaging Client, der eine Vielzahl von Chat-Protokollen unterst�tzt (ICQ, XMPP, AIM, mit Plugins auch z.B. Skype und Twitter). 

F�r Pidgin ist ein OTR-Plugin verwendbar, dass es erm�glicht, jede mit Pidgin gef�hrte Unterhaltung zu verschl�sseln, wenn das Gegen�ber auch OTR versteht. Installieren l�sst sich das Plugin einfach �ber das Plugin-Men�\sidenote{Werkzeuge - Plugins}. Alternativ kann es �ber die OTR-Website heruntergeladen und mit dem Installer (unter Windows) installiert werden\sidenote{http://www.cypherpunks.ca/otr/}. 

Andere Instant Messaging Clients, die OTR unterst�tzen sind Adium (Mac OS X), Gajim und Gitterbot (Android-Ger�te).
 
Ein Browser-Plugin, dass es erm�glicht, OTR-verschl�sselte Unterhaltungen mit mehreren Teilnehmern gleichzeitig in einem Chat-Raum zu f�hren, ist Cryptocat\sidenote{https://crypto.cat}. Nach Installation des Plugins lassen sich einfach Chatr�ume erstellen, denen jeder ohne Registrierung beitreten kann, um dann sicher zu kommunizieren.

\newthought{Ein Kommentar zum Schluss:} All der Aufwand lohnt sich nicht, wenn die Mitschnitte, der via OTR verschl�sselten Gespr�che im Klartext auf der Festplatte hinterlegt werden, auf dem eigenen Rechner oder dem des Gespr�chspartners. Also Gespr�chsmitschnitte ausschalten oder auch verschl�sseln!
\end{document}
